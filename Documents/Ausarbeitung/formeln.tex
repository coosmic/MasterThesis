%\documentclass[sigconf, titlepage, twoside]{acmart}
\documentclass[12pt,titlepage, twoside]{article}
%\usepackage{titlesec}

% language stuff
\usepackage{german}           % deutsche Überschriften etc.
\usepackage{ziffer}           % For , instead of . as Decimal Separator
\usepackage[utf8]{inputenc} % direkte Eingabe von Umlauten

% Layout-Einstellungen
\usepackage{parskip}          % Abstand statt Einrückung
\frenchspacing                % no extra space after periods
\usepackage{parskip}          % paragraph gaps instead of indentation
\usepackage{times}            % default font Times
\tolerance=9000               % avoid words across right border

% miscellaneous
\usepackage{graphicx}         % graphics
\usepackage{subcaption}       % subfigures
\usepackage{hhline}           % double lines in tables
\usepackage{amsfonts}         % real numbers etc.
\usepackage[rightcaption]{sidecap} % figure captions on the right (optional)
\usepackage{hyperref}         % for URLs
\usepackage{listings}         % for code samples
\usepackage{fancyhdr}         % for header line
\usepackage{lastpage}         % for last page count

\usepackage{makecell}

\usepackage{xcolor}           % Code highligting

\usepackage{amsmath}

\colorlet{punct}{red!60!black}
\definecolor{background}{HTML}{EEEEEE}
\definecolor{delim}{RGB}{20,105,176}
\definecolor{string}{RGB}{0,0,255}
\colorlet{numb}{magenta!60!black}

\lstdefinelanguage{json}{
    basicstyle=\normalfont\ttfamily,
    %numbers=left,
    %numberstyle=\scriptsize,
    stepnumber=1,
    numbersep=8pt,
    showstringspaces=false,
    breaklines=true,
    frame=lines,
    stringstyle=\ttfamily\color{string},
    %backgroundcolor=\color{background},
    literate=
     *{0}{{{\color{numb}0}}}{1}
      {1}{{{\color{numb}1}}}{1}
      {2}{{{\color{numb}2}}}{1}
      {3}{{{\color{numb}3}}}{1}
      {4}{{{\color{numb}4}}}{1}
      {5}{{{\color{numb}5}}}{1}
      {6}{{{\color{numb}6}}}{1}
      {7}{{{\color{numb}7}}}{1}
      {8}{{{\color{numb}8}}}{1}
      {9}{{{\color{numb}9}}}{1}
      {:}{{{\color{punct}{:}}}}{1}
      {,}{{{\color{punct}{,}}}}{1}
      {\{}{{{\color{delim}{\{}}}}{1}
      {\}}{{{\color{delim}{\}}}}}{1}
      {[}{{{\color{delim}{[}}}}{1}
      {]}{{{\color{delim}{]}}}}{1},
}
%
%\newcommand{\imageSizeTwo}{0.49\textwidth}
%\newcommand{\imageSizeTwoHeight}{7.5cm}
\newcommand{\imageSizeTwo}{0.35\textwidth}
\newcommand{\imageSizeTwoHeight}{4.5cm}
\newcommand{\imageSizeThree}{0.3\textwidth}
\newcommand{\imageSizeThreeHeight}{5cm}
\newcommand{\imageWidthFour}{width=2cm height=2cm}

% Hier bei Bedarf die Seitenränder einstellen
\usepackage{geometry}
%\geometry{a4paper}
\geometry{a4paper, top=3.5cm, bottom=2.5cm} 

% Kopf- und Fußzeile
\fancyhead{} % clear all header fields
\fancyhead[RO,LE]{\leftmark}

%%%%%%%%%%%%%%%%%%%%%%%%%%%%%%%%%%%%%%%%%%%%%%%%%%%%%%%%%%%%%%
\begin{document}

\section{Stand der Technik}
%-------------------------------------------------------------
\label{sec:stand}

\subsection{SURF}

Integral-Bild $I_\Sigma$ (Gleichung \ref{eq:sfm:surf:integral}) aus den Intensitätswerten $I$ des Bildes berechnet.
\begin{equation}
    \label{eq:sfm:surf:integral}
    I_\Sigma (x,y) = \sum_{i=0}^{i\leq x}\sum_{j=0}^{j\leq y}I(i,j)
\end{equation}

Summe der Intensitätswerte

\begin{equation}
    \label{eq:sfm:surf:f}
    f(D) = I_\Sigma(x_{P_1}, y_{P_1}) + I_\Sigma(x_{P_4}, y_{P_4}) - I_\Sigma(x_{P_2}, y_{P_2}) - I_\Sigma(x_{P_3}, y_{P_3})
\end{equation}

Approximierte Hesse-Matrix für einen Punkt $p$ 

mit der Skalierung $s$

\begin{equation}
    \label{eq:sfm:surf:h}
    H_{approx.}(p,s) = \left( \begin{smallmatrix} D_{xx}(p,s)&D_{xy}(p,s)\\ D_{xy}(p,s)&D_{yy}(p,s) \end{smallmatrix} \right)
\end{equation}

Die Determinante der Hesse-Matrix

mit $w\approx 0,9$
\begin{equation}
    \label{eq:sfm:surf:det}
    det(H_{approx.}) = D_{xx}D_{yy}-(wD_{xy})^2
\end{equation}

\subsection{ORB}

Momente

\begin{equation}
    \label{eq:sfm:orb:m}
    m_{pq} = \sum_{x,y}{x^py^qI(x,y)}
\end{equation}

Massezentrum $C$ des Bildes 

\begin{equation}
    \label{eq:sfm:orb:c}
    C = ( \frac{m_{10}}{m_{00}}, \frac{m_{01}}{m_{00}} )
\end{equation}

Orientierung $\theta$ des Bildes

\begin{equation}
    \label{eq:sfm:orb:theta}
    \theta = atan2(m_{01},m_{10})
\end{equation}

\subsection{SfM}

Fundamental-Matrix $F=K_1^{-1T}EK_2^{-1}$

$K_1$ und $K_2$ sind die beiden Kalibrierungs-Matrizen

$E$ ist die Essential-Matrix

Matrix A bilden

\begin{equation}
    Af=0
\end{equation}

Einzelne Zeile

\begin{equation}
    p_{Ai}^TFp_{Bi}=0
\end{equation}

Einzelne Zeile ausgeschrieben

\begin{equation}
    \label{eq:sfm:ax:line}
    \begin{aligned}
    x_{p_{Ai}}x_{p_{Bi}}f_1 + x_{p_{Ai}}y_{p_{Bi}}f_2 + x_{p_{Ai}}f_3 + &\\
    y_{p_{Ai}}x_{p_{Bi}}f_4 + y_{p_{Ai}}y_{p_{Bi}}f_5 + y_{p_{Ai}}f_6 + &\\
    x_{p_{Bi}}f_7 + y_{p_{Bi}}f_8 + f9 &= 0
    \end{aligned}
\end{equation}

Das Gleichungssystem wird mittels Singulärwertzerlegung (SVD) gelöst. Man erhält drei Matrizen $U$, $S$ und $V$. 

Approximation $\hat{F}$ von $F$ bilden

\begin{equation}
    \label{eq:sfm:aprox:f2}
    V[9] = \begin{bmatrix}
        \hat{f_1} & \hat{f_2} & \hat{f_3} & \hat{f_4} & \hat{f_5} & \hat{f_6} & \hat{f_7} & \hat{f_8} & \hat{f_9}
    \end{bmatrix}
\end{equation}
\begin{equation}
    \label{eq:sfm:aprox:f}
    \hat{F}=\begin{bmatrix}
        \hat{f_1} & \hat{f_2} & \hat{f_3}\\
        \hat{f_4} & \hat{f_5} & \hat{f_6}\\
        \hat{f_7} & \hat{f_8} & \hat{f_9}
    \end{bmatrix}
\end{equation}

Korrektur von $\hat{F}$

\begin{equation}
    \label{eq:sfm:camera:pose:w}
    \hat{S}=\begin{bmatrix}
    s_1 & -1 & 0\\
    1 & s_2 & 0\\
    0 & 0 & 0
\end{bmatrix}
\end{equation}
\begin{equation}
    \label{eq:sfm:bundle:adjustment}
    \hat{F} = U\hat{S}V^T
\end{equation}

Berechnung von $\hat{E}$

\begin{equation}
    \label{eq:sfm:camera:pose:w}
    D=\begin{bmatrix}
    s_1 & -1 & 0\\
    1 & s_2 & 0\\
    0 & 0 & c
\end{bmatrix}
\end{equation}
\begin{equation}
    \label{eq:sfm:bundle:adjustment}
    \hat{E} = UDV^T
\end{equation}

Schätzung von Kamera-Position und Ausrichtung:

\begin{equation}
    \label{eq:sfm:camera:pose:w}
W=\begin{bmatrix}
    0 & -1 & 0\\
    1 & 0 & 0\\
    0 & 0 & 1
\end{bmatrix}
\end{equation}
\begin{equation}
    \label{eq:sfm:camera:pose}
    \begin{split}
    c_1=U(:,3),&\qquad R_1=UWV^T \\
    c_2=-U(:,3),&\qquad R_2=UWV^T \\
    c_3=U(:,3),&\qquad R_3=UW^TV^T \\
    c_4=-U(:,3),&\qquad R_4=UW^TV^T
    \end{split}
\end{equation}

Bundle-Adjustment

\begin{equation}
    \label{eq:sfm:bundle:adjustment}
    \min_{a_j, P_i} = \sum_{i=1}^n \sum_{j=1}^m v_{ij} d(Q(a_j, P_i), x_{ij})
\end{equation}

Hierbei gilt $v_{ij} = 1$, wenn der Punkt $i$ von Kamera $j$ erfasst wurde, ansonsten $0$.
$Q(a_j, P_i)$ ist die geschätzte Projektion des Punktes $P_i$ auf das Bild der Kamera $j$, und $d$ ist die Euklidische Distanz. $x_{ij}$ ist der detektierte Punkt auf dem Bild.

\subsection{Registrierung von 3D-Punktwolken}
\label{sec:stand:registrierung}

Registrierung als Optimierungsproblem

\begin{equation}
    \label{eq:icp}
    \underset{R,t}{\operatorname{argmin}}(\sum_{i=1}^N{\|Rp_{s_i} + t - p_{t_i}\|^2})
\end{equation}
Hierbei sind $R$ und $t$ die gesuchte Rotation und Translation und $p_{t_i}$ und $p_{s_i}$ bilden ein korrespondierendes Punktpaar aus den beiden zu registrierenden Punktwolken. 
$N$ ist die Anzahl Punkte in der Quell-Punktwolke.

Massezentrum Punktwolke:

\begin{equation}
    \label{eq:icp:mean}
    c = \frac{1}{n}\sum_{i=1}^n{p(i)}
\end{equation}

Kreuzkovarianz-Matrix $H$ 

\begin{equation}
    \label{eq:icp:cross:cov}
    H = \sum_{i=1}^n{(p_{t_i} - c_t)(p_{s_i} - c_s)^T}
\end{equation}

SVD

\begin{equation}
    \label{eq:icp:svd}
    svd(H) = UDV^T
\end{equation}

ICP Rotation:

\begin{equation}
    \label{eq:icp:rotation}
    R = VU^T
\end{equation}

ICP Translation:

\begin{equation}
    \label{eq:icp:translation}
    t=c_t-Rc_s
\end{equation}

Skalierung:

Ist $R$ bekannt, können die Vektoren $s$ und $t$ gebildet werden (Siehe Gleichung \ref{eq:icp_scale:1}).

\begin{equation}
    \label{eq:icp_scale:1}
    s_i = R (p_{s_i} - \bar{p_s}),\quad t_j = p_{t_j} - \bar{p_t}
\end{equation}

Mittels dieser Vektoren kann die Skalierung $\hat{s}$ berechnet werden. Die Berechnung ist in Gleichung \ref{eq:icp_scale:2} zu sehen.

\begin{equation}
    \label{eq:icp_scale:2}
    \hat{s} = \sum_{i,j}{t_j^Ts_i} / \sum_{i}{s_i^Ts_i}
\end{equation}

Die Translation muss auch in angepasster Form berechnet werden, wie in Gleichung \ref{eq:icp_scale:3} zu sehen ist.

\begin{equation}
    \label{eq:icp_scale:3}
    t = \bar{p_t} - \hat{s}  R \bar{p_s}
\end{equation}

RPM-Net

Optimierungsproblem RPM

\begin{equation}
    \label{eq:rpm:opt}
    \underset{M,R,t}{\operatorname{argmin}}(\sum_{i=1}^N\sum_{k=1}^K{m_{jk}(\|Rp_{s_i} + t - p_{t_k}\|_2^2-\alpha)})
\end{equation}

Gewichtungsmatrix $M$ für die Zuordnungen zwischen den Quell- und Zielpunkten. 

Der Parameter $\alpha$ ist dazu da, um Korrespondenzen zwischen schlechten Paaren zu bestrafen. 

RPM-Initialisierung

\begin{equation}
    \label{eq:rpm:m}
    m_{ij} = e^{-\beta(\|Rp_{s_i} + t - p_{t_j}\|_2^2-\alpha)}
\end{equation}

RPM-Net-Initialisierung

\begin{equation}
    \label{eq:rpm:net:m}
    m_{ij} = e^{-\beta(\|\hat{f}_{s_i} - f_{t_j}\|_2^2-\alpha)}
\end{equation}
$\hat{f}_{s_i}$ ist der hybride Feature-Vektor für den aus der vorherigen Iteration transformierten Punkt $p_{s_i}$ und $f_{t_j}$ ist der Feature-Vektor für den Punkt $p_{t_j}$.

\newpage
\section{Realisierung}

\subsection{Umsetzung Segmentierung}
%-------------------------------------------------------------
\label{sec:realisierung:implementierung3}

Abbildungen $m_j$

\begin{equation}
\label{eq:hauptkruemmung:1}
m_j = (I - n_i \otimes n_i ) \cdot n_j
\end{equation}

 Kovarianzmatrix $C_i$

\begin{equation}
\label{eq:hauptkruemmung:2}
C_i = \frac{1}{k} \sum_{j=1}^k{(m_j - \bar{m}) \otimes (m_j - \bar{m})}
\end{equation}

Die Hauptkrümmung kann aus den Eigenwerten $0 \leq \lambda_1 \leq \lambda_2 \leq \lambda_3$ von $C_i$ bestimmt werden. 

Funktionen für f:

\begin{equation}
\label{eq:hauptkruemmung:3}
\begin{array}{ll}
k_1(p_i) = \lambda_3 \\
k_2(p_i) = \lambda_2 \\
k_3(p_i) = (\lambda_3 + \lambda_2) / 2 
\end{array}{}
\end{equation}

Entscheidungsregel

\begin{equation}
\label{eq:manuell_classifier}
f(p_i) = \left\{
\begin{array}{ll}
1 & k(p_i) \geq T_k \\
0 & \, \textrm{sonst} \\
\end{array}
\right. 
\end{equation}

Verbesserung des Segmentierungs Ergebnisses

\begin{equation}
\label{eq:improve_score}
\begin{array}{l}
L =  \{0,1,2\}\\
x \in L\\
w(x) = \left\{
\begin{array}{ll}
0,5 & x = 2 \\
1 & \, \textrm{sonst} 
\end{array}
\right.\\ 
H_i(x) = \sum_{j=0}^{k}{(w(x) | N_{ij} = x)}\\
s_i = argmax_x(H_i(x))
\end{array}
\end{equation}

\newpage
\section{Ergebnisse}
%-------------------------------------------------------------
\label{sec:ergebnisse}

\subsection{Vergleich von Verfahren zur Segmentierung von Pflanzen auf 3D-Punktwolken}

\textbf{Datenbasis für das Training von PointNet++}

\begin{equation}
    \label{eq:hauptkruemmung:2}
    j_c(\hat{P_c},P_c) = \frac{|\hat{P_c} \cap P_c|}{|\hat{P_c} \cup P_c|}
\end{equation}

$\hat{P_c}$ ist die Menge an Punkten aus der Schätzung des Netzes, welche als $C$ klassifiziert werden. 

$P_c$ ist die Menge der als $C$ klassifizierten Punkte aus dem Groundtruth.

\end{document}